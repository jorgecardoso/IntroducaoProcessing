\chapter{Comentário Final (Documentação)}

Um programa não é apenas formado pelo conjunto de instruções da linguagem de programação. Se assim fosse, a maioria dos programas seria ininteligíveis para nós. Atentem no programa do Exemplo [x]. O que faz esse programa? Como o faz? Como o modificamos para o adaptar a outra situação? Torna-se difícil (embora não impossível) responder a estas perguntas, porque o programa não está documentado. Agora observem o Exemplo [xpto].

\section{Comentários}
Os comentários são pedaços de texto que não afectam a execução do programa e que 
servem para documentar o código, explicando o funcionamento do programa.

Os comentários devem ser usados para explicar secções de código cujo significado não
seja óbvio\footnote{O significado é óbvio ou não dependendo do contexto.}.

De forma a que o compilador possa distinguir entre texto de comentários e texto
do programa é necessário que os comentários sejam identificados por símbolos especiais.
Existem dois tipos de comentários:
\begin{description}
\item[Linha] Os comentários de linha são identificados pelos caracteres ``\texttt{//}''. 
Todo o texto seguinte na linha é considerado comentário e, portanto, ignorado. Exemplo:
\begin{lstlisting}
int altura[100]; // a altura de 100 pessoas em centímetros
\end{lstlisting}
\item[Bloco] Os comentários de bloco servem para definir um comentário que se prolonga por
várias linhas. Os comentários de bloco são delimitados por duas sequências de caracteres:
``\texttt{/*}'' e ``\texttt{*/}''. Estes comentários permitem-nos dar uma explicação mais
longa sobre o código. Exemplo:
\begin{lstlisting}
/* Determinar a altura máxima e mínima do grupo de pessoas.
   Precisamos de variáveis temporárias para armazenar as 
   alturas miníma e máxima enquanto estamos a percorrer o 
   vector. */
   int minimo = 0, maximo = 10000;
   int i;
   for (i = 0; i < 100; i = i + 1) {
       if (altura[i] < minimo) {           
           minimo = altura[i];
       } else if (altura[i] > maximo) {
           maximo = altura[i];
       }
   }
   print("Altura máxima: " + maximo);
   print("Altura miníma: " + minimo);
\end{lstlisting}
\end{description}

A utilização de um tipo de comentário ou outro é, maioritariamente, uma questão de estilo. 
Podemos utilizar indiferencialmente um ou outro. Os exemplos anteriores poderiam ter sido escritos como:
\begin{lstlisting}
inteiro altura[100]; /* A altura de 100 pessoas em centímetros */

// Determinar a altura máxima e mínima do grupo de pessoas.
// Precisamos de variáveis temporárias para armazenar as 
// alturas miníma e máxima enquanto estamos a percorrer o 
// vector.
   int minimo = 0, maximo = 10000;
   int i;
   for (i = 0; i < 100; i = i + 1) {
       if (altura[i] < minimo) {           
           minimo = altura[i];
       } else if (altura[i] > maximo) {
           maximo = altura[i];
       }
   }
   print("Altura máxima: " + maximo);
   print("Altura miníma: " + minimo);
\end{lstlisting}


\section{Exercícios}
\begin{enumerate}
\item 
Documente, se ainda não o fez, os últimos dois programas que fez.	
\end{enumerate}